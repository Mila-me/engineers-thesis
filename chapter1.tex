% !TeX spellcheck = pl_PL
% !TeX encoding = UTF-8
\chapter{Wstęp}
\section{Tło}
Podstawą dla rozwoju społeczeństwa niewątpliwie jest przekazywanie wiedzy, ale jaką możliwość przekazywania wiedzy mielibyśmy bez komunikacji.
Najnowsze badania\cite{doi:10.1126/sciadv.aaw3916} wskazują, że jej rozwój nastąpił w okresie od 20 milionów lat temu do 200 tys. lat temu. To bardzo dawno temu. Natomiast, telegraf\cite{wiki:telegraf} wynaleziono raptem w XVIII w. Na koniec tego wieku możemy datować początek komunikacji długodystansowej. Wolno, wręcz powolnie i kilkaset lat później mamy już Internet\cite{wiki:internet}, którego początki datujemy na lata 60 XX w. Następnym etapem rozwoju tej pięknej dziedziny jest niewątpliwie zwiększanie przepustowości światłowodów, które następuje skokowo jak i przygotowanie do kolonizacji odległych (na początek tylko w obrębie naszego Układu Słonecznego) planet z którymi oczywiście musimy utrzymywać kontakt.
\section{Cel}
Kontakt, stanowi on sedno niniejszej pracy. Pragnę w jej ramach przygotować, przetestować, opisać i zaprezentować system komunikacji.
Mój system komunikacji ma działać na odległości do 5-10 km w idealnych warunkach, przy wykorzystaniu technologii Lora\cite{enwiki:lora} w paśmie poniżej 1 GHz.
Jest to zadanie wymagające pod względem technicznym, jak i praktycznym. Trzeba zgrać ze sobą różne moduły, przygotować funkcjonalną obudowę, stworzyć aplikację mobilną i napisać kod na mikrokontroler, a na koniec wszystko przetestować i opisać. Cały system musi działać niezawodnie, bo w jaki inny sposób możemy mówić o komunikacji, jeśli nie jest ona niezawodna. Mamy wtedy raptem nie w pełni sprawną protezę kompletnego rozwiązania, czego bardzo nie chcę i poprzez odpowiednie testy udowodnię, że moje rozwiązanie jest kompletne od strony zarówno hardware'u jak i software'u.
\section{Szablon}
Niniejsza praca podzielona jest na następujące rozdziały i podrozdziały:
\begin{enumerate}
	\item Wstęp - omówienie celu jak i zakresu pracy.
	\item Tło teoretyczne - omówienie koncepcji i technologii używanych przez mikrokontroler, moduł lora i moduł bluetooth.
	\item Projekt urządzenia i oprogramowania - zaprezentowanie narzędzi i komponentów wykorzystanych do stworzenia od strony hardware'owej jak i software'owej systemu i jego kluczowych funkcji.
	\item Badanie efektywności - sprawdzenie efektywności zaprezentowanego rozwiązania
	\item Podsumowanie - podsumowanie pracy
	\item Bibliografia
	\item Spisy rysunków, tabel i załączników
\end{enumerate}