% !TeX spellcheck = pl_PL
\newpage % Rozdziały zaczynamy od nowej strony.
\section{Założenia projektowe}
Przy realizacji niniejszej pracy przyjęto kilka założeń:
\begin{itemize}
	\item Przede wszystkim system będzie umożliwiał komunikację między dwoma użytkownikami przy użyciu aplikacji do wprowadzania tekstu na smartfona i dedykowanego urządzenia.
	\item Komunikacja będzie odbywać się przy wykorzystaniu standardu Lora.
	\item System będzie zaprojektowany z myślą o niskim poborze prądu, dzięki czemu będzie mógł być zasilany z baterii lub powerbanku.
	\item Obudowa do urządzenia zewnętrznego będzie wykonana w technologii druku 3d.
	\item Całość lub jeżeli to niemożliwe, to jak największa część systemu komunikacji zostanie opublikowana na otwartoźródłowej licencji.
\end{itemize}
\subsection{Koncepcja wykonania}
Pracę rozpoczęto od rozważenia różnych mikrokontrolerów, względy praktyczne jak i dostępność zadecydowały o wyborze Rasberry Pico (RP2040). Niestety powyższy mikrokontroler pozbawiony jest możliwości komunikacji bezprzewodowej, co też skutkuje tym, że trzeba do niego dobrać odpowiedni moduł Bluetooth, w tym wypadku hm-06, komunikujący się z mikro-kontrolerem przy użyciu UART\footnote{Uniwersalny asynchroniczny nadajnik-odbiornik.}
Jeśli zaś chodzi o Lora, to możliwość komunikacji przy użyciu tego protokołu uzyskano dzięki użyciu odpowiedniej przystawki - SB Components SKU21628.