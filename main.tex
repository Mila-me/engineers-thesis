%%%%%%%%%%%%%%%%%%%%%%%%%%%%%%%%%%%%%%%%%%%%%%%%%%%%%%%
%% Bachelor's & Master's Thesis Template             %%
%% Copyleft by Artur M. Brodzki & Piotr Woźniak      %%
%% Faculty of Electronics and Information Technology %%
%% Warsaw University of Technology, 2019-2020        %%
%%%%%%%%%%%%%%%%%%%%%%%%%%%%%%%%%%%%%%%%%%%%%%%%%%%%%%%

\documentclass[
    left=2.5cm,         % Sadly, generic margin parameter
    right=2.5cm,        % doesnt't work, as it is
    top=2.5cm,          % superseded by more specific
    bottom=3cm,         % left...bottom parameters.
    bindingoffset=6mm,  % Optional binding offset.
    nohyphenation=false % You may turn off hyphenation, if don't like.
]{eiti/eiti-thesis}

\langpol % Dla języka angielskiego mamy \langeng
\graphicspath{{img/}}             % Katalog z obrazkami.
\addbibresource{bibliografia.bib} % Plik .bib z bibliografią

\begin{document}

%--------------------------------------
% Strona tytułowa
%--------------------------------------
\EngineerThesis % Dla pracy inżynierskiej mamy \EngineerThesis
\instytut{Mikroelektroniki i Optoelektroniki}
\kierunek{Elektronika i Telekomunikacja}
\specjalnosc{Inżynieria Komputerowa}
\title{
    Dedykowany system komunikacji z wykorzystaniem \\
    protokołu Lora
}
\engtitle{ % Tytuł po angielsku do angielskiego streszczenia
    Dedicated communication system using Lora protocol
}
\author{{Emil Michalik}}
\album{280293}
\promotor{dr inż. Marek Niewiński}
\date{\the\year}
\maketitle

%--------------------------------------
% Streszczenie po polsku
%--------------------------------------
\cleardoublepage % Zaczynamy od nieparzystej strony
\streszczenie
Niniejsza praca dotyczy konstrukcji dedykowanego systemu komunikacji z wykorzystaniem protokołu Lora.
Do zakresu pracy należało wykonanie oraz przetestowanie systemu, który służył do komunikacji przy użyciu aplikacji na smartfon z systemem Android - napisanej w języku Kotlin, jak i urządzenia zewnętrznego czyli mikrokontrolera w dedykowanej obudowie wraz z modułem dającym możliwość do komunikacji przy użyciu protokołu Lora, w paśmie poniżej 1 GHz.
\slowakluczowe raspberry pico, lora, mikrokontroler, android, kotlin, micropython

%--------------------------------------
% Streszczenie po angielsku
%--------------------------------------
\newpage
\abstract
This Engineering Thesis concers the construction of a dedicated communication system using the Lora protocol.
The scope of work refers to building and testing a system, which enables user to communicate using application on the smartphone with Android operating system - written in Kotlin programming language, and also device in dedicated case with module which allows to communicate using the Lora protocol in the frequency lower than 1 Ghz.
\keywords raspberry pico, lora, microcontroller, android, kotlin, micropython

%--------------------------------------
% Oświadczenie o autorstwie
%--------------------------------------
\cleardoublepage  % Zaczynamy od nieparzystej strony
\pagestyle{plain}
\makeauthorship

%--------------------------------------
% Spis treści
%--------------------------------------
\cleardoublepage % Zaczynamy od nieparzystej strony
\tableofcontents

%--------------------------------------
% Rozdziały
%--------------------------------------
\cleardoublepage % Zaczynamy od nieparzystej strony
\pagestyle{headings}

\newpage % Rozdziały zaczynamy od nowej strony.
\section{Wstęp}
Zagadnieniem systemów komunikacji człowiek zajmuje się już od XIX w. Rozwój systemów komunikacji wiąże się, z rozwojem wielu dziedzin techniki i nauki na niespotykaną do tej pory skalę. Problemy komunikacji na odległość i ich rozwiązania zajmują najtęższe umysły od prawie 200 lat. Przykładem a zarazem pierwszym urządzeniem komunikacyjnym jest telegraf\cite{wiki:telegraf2023}.
W odpowiedzi na potrzeby i różnego rodzaju wymagania powstały liczne dedykowane systemy komunikacyjne i urządzenia z nich korzystające.
Większość obecnie dostępnych systemów komunikacji wymaga dostępu do jakiejś sieci np. GSM(telefonia komórkowa, w tym MMS-y i SMS-y), Internet(popularne komunikatory takie jak np. Whatsapp, Signal lub Telegram).
Istotnym więc dla tej pracy dyplomowej stało się wytworzenie systemu komunikacyjnego, który działał by niezależnie od dostępności do [jakiejkolwiek] sieci - co czyniłoby go idealnym rozwiązaniem w sytuacjach kryzysowych.
Wytworzone rozwiązanie oparte jest w dużej mierze na rozwiązaniach otwartych, co umożliwia jego dostosowanie do szczególnych wymogów.
\begin{figure}[!h]
	% Znacznik \caption oprócz podpisu służy również do wygenerowania numeru obrazka;
	\caption{Schemat telegrafu}
	% dlatego zawsze pamiętaj używać najpierw \caption, a potem \label
	\label{fig:telegraf}
	% Zamiast width można też użyć height, etc. 
	\centering \includegraphics[width=0.5\linewidth]{telegraf.png}
\end{figure}
\subsection{Cele pracy}
Do celów niniejszej pracy należy:
\begin{itemize}
	\item zaprojektowanie, wykonanie i przetestowanie systemu do komunikacji tekstowej wykorzystującej protokół Lora,
	\item przygotowanie dedykowanej aplikacji mobilnej do obsługi tego systemu,
	\item przygotowanie dokumentacji technicznej,
	\item zaprojektowanie i montaż dedykowanej obudowy do części systemu wykorzystującej mikrokontroler,
	\item zbadanie efektywności stworzonego rozwiązania (ze względu na maksymalny dystans, przy którym komunikacja będzie jeszcze możliwa)         % Wygodnie jest trzymać każdy rozdział w osobnym pliku.
\newpage % Rozdziały zaczynamy od nowej strony.
\section{Standard komunikacji Lora}
\subsection{Wiadomości ogólne}
\subsection{Rozwój standardu}
\subsection{Specyfikacje standardu}
\subsection{Przykładowe zastosowania standardu} 
%\input{tex/2-de-finibus}    % Umożliwia to również łatwą migrację do nowej wersji szablonu:
%\input{tex/3-code-listings} % wystarczy podmienić swoje pliki main.tex i eiti-thesis.cls
                            % na nowe wersje, a cały tekst pracy pozostaje nienaruszony.
% !TeX spellcheck = pl_PL
\newpage % Rozdziały zaczynamy od nowej strony.
\section{Standard Bluetooth}
\section{Wstęp}
Standard Bluetooth liczy ponad 1000 stron, wobec czego zamieszczenie go w całości w tej pracy dyplomowej byłoby niemożliwe i/lub bardzo trudne, toteż załączam tylko wybrane fragmenty.
% !TeX spellcheck = pl_PL
\newpage % Rozdziały zaczynamy od nowej strony.
\section{Założenia projektowe}
Przy realizacji niniejszej pracy przyjęto kilka założeń:
\begin{itemize}
	\item Przede wszystkim system będzie umożliwiał komunikację między dwoma użytkownikami przy użyciu aplikacji do wprowadzania tekstu na smartfona i dedykowanego urządzenia.
	\item Komunikacja będzie odbywać się przy wykorzystaniu standardu Lora.
	\item System będzie zaprojektowany z myślą o niskim poborze prądu, dzięki czemu będzie mógł być zasilany z baterii lub powerbanku.
	\item Obudowa do urządzenia zewnętrznego będzie wykonana w technologii druku 3d.
	\item Całość lub jeżeli to niemożliwe, to jak największa część systemu komunikacji zostanie opublikowana na otwartoźródłowej licencji.
\end{itemize}
\subsection{Koncepcja wykonania}
Pracę rozpoczęto od rozważenia różnych mikrokontrolerów, względy praktyczne jak i dostępność zadecydowały o wyborze Rasberry Pico (RP2040). Niestety powyższy mikrokontroler pozbawiony jest możliwości komunikacji bezprzewodowej, co też skutkuje tym, że trzeba do niego dobrać odpowiedni moduł Bluetooth, w tym wypadku hm-06, komunikujący się z mikro-kontrolerem przy użyciu UART\footnote{Uniwersalny asynchroniczny nadajnik-odbiornik.}
Jeśli zaś chodzi o Lora, to możliwość komunikacji przy użyciu tego protokołu uzyskano dzięki użyciu odpowiedniej przystawki - SB Components SKU21628.
% !TeX spellcheck = pl_PL
\newpage % Rozdziały zaczynamy od nowej strony.
\section{Realizacja}
\subsection{Konstrukcja dedykowanego urządzenia}
\subsection{Zasilanie}
% !TeX spellcheck = pl_PL
\newpage % Rozdziały zaczynamy od nowej strony.
\section{Testowanie}
\subsection{Prototyp testowy dedykowanego urządzenia}
\subsection{Prototyp aplikacji na Androida}
\subsection{Pomiar odległości}
% !TeX spellcheck = pl_PL
\newpage % Rozdziały zaczynamy od nowej strony.
\section{Podsumowanie}
%\newpage % Rozdziały zaczynamy od nowej strony
%\section{Summatio}          % Można też pisać rozdziały w jednym pliku.
%\lipsum[5-10]

%--------------------------------------------
% Literatura
%--------------------------------------------
\cleardoublepage % Zaczynamy od nieparzystej strony
\printbibliography

%--------------------------------------------
% Spisy (opcjonalne)
%--------------------------------------------
\newpage
\pagestyle{plain}

% Wykaz symboli i skrótów.
% Pamiętaj, żeby posortować symbole alfabetycznie
% we własnym zakresie. Ponieważ mało kto używa takiego wykazu,
% uznałem, że robienie automatycznie sortowanej listy
% na poziomie LaTeXa to za duży overkill.
% Makro \acronymlist generuje właściwy tytuł sekcji,
% w zależności od języka.
% Makro \acronym dodaje skrót/symbol do listy,
% zapewniając podstawowe formatowanie.
% //AB
\vspace{0.8cm}
\acronymlist
\acronym{EiTI}{Wydział Elektroniki i Technik Informacyjnych}
\acronym{PW}{Politechnika Warszawska}

\listoffigurestoc     % Spis rysunków.
\vspace{1cm}          % vertical space
\listoftablestoc      % Spis tabel.
\vspace{1cm}          % vertical space
\listofappendicestoc  % Spis załączników

% Załączniki
\newpage
\appendix{Nazwa załącznika 1}
\lipsum[1-8]

\newpage
\appendix{Nazwa załącznika 2}
\lipsum[1-4]

% Używając powyższych spisów jako szablonu,
% możesz tu dodać swój własny wykaz bądź listę,
% np. spis algorytmów.

\end{document} % Dobranoc.
