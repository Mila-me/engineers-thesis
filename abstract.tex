% !TeX encoding = UTF-8
% !TeX spellcheck = pl_PL
    % Streszczenie
    \cleardoublepage
    \vspace*{2\baselineskip}
    \begin{center}
	{\large\bfseries Streszczenie}\par\bigskip\end{center}
 \noindent{\textbf {Tytuł}}: \tytulpl
 \par
    \vspace*{1\baselineskip}
    {
    %%%%%%%%%%%%%%%%%%%%%%%%%%%%%%%%%%%%%%%%%%%%%%%%%%%%%%%%%%%%%%%%%%%%%%%%%%%%%%%%%%%%%%%%%
    %%%%%%%%%%%%%%%%%%%%%%%%%%%%%%%%%%%%%%%%%%%%%%%%%%%%%%%%%%%%%%%%%%%%%%%%%%%%%%%%%%%%%%%%%
  	Obecnie dostępne systemy komunikacji wykorzystują w dużej mierze Internet, a co w sytuacji gdy go zabraknie?
	Na to pytanie postaram się odpowiedzieć w mojej pracy.

	W mojej pracy starałem się przybliżyć różne aspekty konstruowania, wykonania, bądź działania systemów komunikacji.
	Ponadto opisałem działanie gotowanie systemu mojego autorstwa

	Mój system składa się (od strony) oprogramowania z dwóch elementów: aplikacji na mikrokontroler i aplikacji mobilnej napisanej na urządzenia z systemem operacyjnym Android.

	Natomiast od strony sprzetowej jest to mikrokontroler Raspberry Pi Pico (RP2040), moduł Bluetooth i moduł do komunikacji na częstotliwości sub-1 Ghz

	Wiele kwestii dotyczących budowy takich - dedykowanych systemów komunikacji jest kwestią umowną. Ja natomiast postarałem się skupić na sprawdzonych, powiedziałbym nawet kuloodpornych rozwiązaniach.

	Słowem końcowym, nie możemy zapomnieć o sytuacji geopolitycznej, obecne działania na terytorium Ukrainy pokazują nam jak ważna jest komunikacja (chociażby na względnie krótki dystans), a moim osobistym życzeniem jest zapewnienie do niej dostępu wszystkim, w miarę przystępnej realizacji.
    %%%%%%%%%%%%%%%%%%%%%%%%%%%%%%%%%%%%%%%%%%%%%%%%%%%%%%%%%%%%%%%%%%%%%%%%%%%%%%%%%%%%%%%%%
    %%%%%%%%%%%%%%%%%%%%%%%%%%%%%%%%%%%%%%%%%%%%%%%%%%%%%%%%%%%%%%%%%%%%%%%%%%%%%%%%%%%%%%%%%
    }
    \vspace*{1\baselineskip}

    \noindent{\textbf {Słowa kluczowe}}: {\itshape Android, raspberry pico, mikrokontroler, lora}
    \par 
   \cleardoublepage
    \begin{center}{\large\bfseries Abstract}\par\bigskip\end{center}
    \noindent{\textbf {Title}}: \tytulen
    \par
    \vspace*{1\baselineskip}
    {
    %%%%%%%%%%%%%%%%%%%%%%%%%%%%%%%%%%%%%%%%%%%%%%%%%%%%%%%%%%%%%%%%%%%%%%%%%%%%%%%%%%%%%%%%%
    %%%%%%%%%%%%%%%%%%%%%%%%%%%%%%%%%%%%%%%%%%%%%%%%%%%%%%%%%%%%%%%%%%%%%%%%%%%%%%%%%%%%%%%%%
    Currently available communication system are using Internet, to the great externs. What if there is no Internet?
    I am going to answer to this question in my thesis.

	In my work, I have tried to introduce various aspects of the construction, implementation, or operation of communication systems.

	My system is composed (from the side of) software of two components: application working on microcontroller and mobile application written to operate on devices with Android operation system.

	On the hardware side, however, it is a raspberry pico microcontroller, a Bluetooth module and a module for communication at sub-1 Ghz frequency

	Many issues concerning the construction of such - dedicated communication systems are a matter of convention. I, however, have tried to focus on proven, I would even say bulletproof solutions.

	In a final word, we must not forget about the geopolitical situation, the current actions on Ukrainian territory show us how important communication is (if only for a relatively short distance), and my personal wish is to provide access to it for all, with a relatively affordable implementation.
    %%%%%%%%%%%%%%%%%%%%%%%%%%%%%%%%%%%%%%%%%%%%%%%%%%%%%%%%%%%%%%%%%%%%%%%%%%%%%%%%%%%%%%%%%
    %%%%%%%%%%%%%%%%%%%%%%%%%%%%%%%%%%%%%%%%%%%%%%%%%%%%%%%%%%%%%%%%%%%%%%%%%%%%%%%%%%%%%%%%%
    }
    \vspace*{1\baselineskip}

    \noindent{\textbf {Keywords}}: {\itshape Android, raspberry pico, microcontroller, lora.}
    
    

    %%%%%%%%%%%%%%%%%%%%%%%%%%%%%%%%%%%%%%%%%%%%%%%%%%%%%%%%%%%%%%%%%%%%%%%%%%%%%%%%%%%%%%%%%
    %%%%%%%%%%%%%%%%%%%%%%%%%%%%%%%%%%%%%%%%%%%%%%%%%%%%%%%%%%%%%%%%%%%%%%%%%%%%%%%%%%%%%%%%%
    %%%%%%%%%%%%%%%%%%%%%%%%%%%%%%%%%%%%%%%%%%%%%%%%%%%%%%%%%%%%%%%%%%%%%%%%%%%%%%%%%%%%%%%%%
    \newpage