\chapter{Kody źródłowe symulacji numerycznych w programie Matlab}

Wszystkie kody źródłowe, z drobnymi zmianami i polonizacją pochodzą stąd: https://www.mathworks.com/help/radar/ug/modeling-the-propagation-of-rf-signals.html

\section{Symulacja na ścieżce w przestrzeni wolnej}
\begin{lstlisting}[language=matlab, xleftmargin=0pt, backgroundcolor={\color{white}}, caption={}, frame=""]
c = physconst('lightspeed');
R0 = [100 1e3 10e3];
freq = (1:1000).'*1e9;
apathloss = fspl(R0,c./freq);
loglog(freq/1e9,apathloss);
grid on;
ylim([90 200]);
legend('Zasięg: 100 m', 'Zasięg: 1 km', 'Zasięg: 10 km','Location','northwest');
xlabel('Częstotliwość (GHz)');
ylabel('Straty na ścieżce (dB)');
title('Straty na ścieżce w wolnej przestrzeni');
\end{lstlisting}

\section{Symulacja na dystansie 5 km podczas opadu deszczu}
\begin{lstlisting}[language=matlab, xleftmargin=0pt, backgroundcolor={\color{white}}, caption={}, frame=""]
R0 = 5e3;                % 5 km zasięg
rainrate = [1 4 20];     % siła opadu w mm/h
el = 0;                  % 0 stopni wznoszenia
tau = 0;                 % polaryzacja horyzontalna

for m = 1:numel(rainrate)
rainloss_itu(:,m) = rainpl(R0,freq,rainrate(m),el,tau)';
rainloss_crane(:,m) = cranerainpl(R0,freq,rainrate(m),el,tau)';
end
loglog(freq/1e9,rainloss_itu);
hold on;
set(gca,'ColorOrderIndex',1); % reset color index for better comparison
loglog(freq/1e9,rainloss_crane,'--');
hold off;
grid on;
legend('Słaby deszcz (ITU)','Średni deszcz (ITU)','Mocny deszcz (ITU)',...
'Słaby deszcz (Crane)','Średni deszcz (Crane)','Mocny deszcz (Crane)', ...
'Location','SouthEast');
xlabel('Częstotliwość (GHz)');
ylabel('Tłumienie na dystansie 5 km (dB)')
title('Tłumienie deszczu dla zadanej polaryzacji');
\end{lstlisting}

\section{Symulacja na dystansie 1 km podczas opadu śniegu}
\begin{lstlisting}[language=matlab, xleftmargin=0pt, backgroundcolor={\color{white}}, caption={}, frame=""]
freq = (1:20)*1e9;
R0 = 1e3;               % 1 km dystansu
snowrate = [0.1 1.5 4]; % odpowiednik opadu deszczu mm/h

for m = 1:numel(snowrate)
snowloss(:,m) = snowpl(R0,freq,snowrate(m));
end
loglog(freq/1e9,snowloss);
grid on;
legend('Lekki śnieg','Średni śnieg','Mocny śnieg', ...
'Location','SouthEast');
xlabel('Częstotliwość (GHz)');
ylabel('Tłumienie na dystansie 1 km (dB)')
title('Tłumienie śniegu');
\end{lstlisting}

\section{Symulacja na dystansie 5 km podczas opadu śniegu}
\begin{lstlisting}[language=matlab, xleftmargin=0pt, backgroundcolor={\color{white}}, caption={}, frame=""]
freq = (1:20)*1e9;
R0 = 5e3;               % 5 km dystansu
snowrate = [0.1 1.5 4]; % odpowiednik opadu deszczu mm/h

for m = 1:numel(snowrate)
snowloss(:,m) = snowpl(R0,freq,snowrate(m));
end
loglog(freq/1e9,snowloss);
grid on;
legend('Lekki śnieg','Średni śnieg','Mocny śnieg', ...
'Location','SouthEast');
xlabel('Częstotliwość (GHz)');
ylabel('Tłumienie na dystansie 1 km (dB)')
title('Tłumienie śniegu');
\end{lstlisting}