% !TeX spellcheck = pl_PL
\newpage % Rozdziały zaczynamy od nowej strony.
\section{Standard komunikacji Lora}
\subsection{Wiadomości ogólne}
Lora\cite{lora} jest techniką modulacji widma rozproszonego wywodzącą się z technologii ang. chirp spread spectrum.
Standard Lora, dalekosiężnej, bezprzewodowej platformy do komunikacji na odległość firmy Semtech, stał się de facto platformą Internetu Rzeczy.
Urządzenia lora i sieci LoRaWAN umożliwiają stawianie czoła wyzwanią takim jak: zarządzanie energiom, redukcja zużycua zasobów naturalnych, kontrola zanieczyszczeń, wydajność infrastruktury, i kontrola w sytuacjach wystąpienia katastrof naturalnych.
\subsection{Rozwój}
Początkowo standard Lora został opracowany przez firmę Cycleo i opatentowany w 2014 roku, następnie firma ta została wykupiona przez Semtech, który zarządza i rozwija standard.
LoRaWAN\footnote{definiuje protokół komunikacji i architekturę systemu. Jest oficjalnym standardem ITU (Międzynarodowej Unii Telekomunikacyjnej), pod nr. ITU-T Y.4480. Tłumaczenie własne.}
Ciągły rozwój protokołu LoRaWAN jest możliwy dzięki organizacji non-profit LoRa Alliance, której to Semtech jest członkiem założycielem.
\subsection{Specyfikacja}
Lora wykorzystuje pasmo nielicencjonowane poniżej 1 GHz (w Europie to: 863 - 870/873 MHz). Technologia ta umożliwia transmisje na duży zasięg z minimalnym poborem energii. Technologia opisuje warstwę fizyczną, podczas gdy inne technologie i protokoły takie jak np. LoRAWAN opisują wyższe warstwy. Umożliwia ona osiągnięcie prędkości transferu danych pomiędzy 0,3 kbit/s, a 27 kbit/s.
\subsection{Przykładowe zastosowania}
W mnogości zastosowań technologii LoRa trudno wymienić zaledwie kilka, zapominając o reszcie. Najlepszy wzgląd na zastosowania daje spojrzenie na przykłady podane przez firmę zarządzającą tą technologią dostępne pod tym linkiem - https://www.semtech.com/lora/resources/lora-white-papers.
W powyższej pracy postarano się wymienić i krótko opisać część zastosowań:
\begin{itemize}
	\item inteligentne mierniki gazu, dające dostęp do wiarygodnych danych i zwiększające bezpieczeństwo infrastruktury krytycznej,
	\item monitorowanie przewożonego ładunku w tzw. zimnym łańcuchu,
	\item zdalne i wytrzymałe systemy pomiarowe, służące poprawie bezpieczeństwa energetycznego,
	\item system monitorowania i detekcji powodzi,
	\item różnorakie systemy tzw. smart homes,
	\item systemy do zarządzania dostępnością wody,
	\item system czujników umożliwiający skrócenie czasu produkcji,
	\item inteligentne pola golfowe,
	\item monitorowanie akustyczne w czasie rzeczywistym,
	\item monitorowanie ogrodów botanicznych,
	\item zarządzanie polami uprawnymi,
	\item inteligentne parkometry,
	\item lokalizowanie skradzionych samochodów i ładunków,
	\item sprawdzanie położenia bydła,
	\item detekcja wycieków radioaktywnych.
\end{itemize}
